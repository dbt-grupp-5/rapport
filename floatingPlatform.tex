\section{Floating Platform}
The initial design for the vehicle was a simple platform with two pontoons, one on either side, and that all other components would be mounted on top of this platform.  The pontoons are constructed from XPS cell foam, which is available in hardware and construction stores. The advantage of XPS over e.g. EPS foam, while more expensive, is that the material is denser and doesn’t absorb water. Cell foam is easy to form and work with, non-hazardous, and offers excellent buoyancy.

As the development progressed, the load that the platform would be required to bear was roughly estimated and served as a basis for the size of the pontoons. They were estimated to displace 20 kg with some margin to the water surface. The platform itself, constructed from glulam wood, is attached to the pontoons with 10 mm threaded rods. An advantage with wood is that it doesn’t sink, is recyclable, cheap, and easy to work with using simple tools. A disadvantage with wood is that it’s heavy. A hollow aluminium construction, e.g. sealed square tubing, could also be buoyant, and offer lighter weight. 

To distribute weight as evenly as possible, the RBR modules placed at either end in the open space between the pontoons, and the rather heavy batteries were placed above the center of each pontoon. The batteries were placed in a slot in the platform to reduce weight, lower center of gravity and increase stability. The available space for electronics was thus limited to the middle of the platform, between the RBR modules.

To improve deflection of head on collisions, a rounded front bumper was added which also serves as a means of adding distance between the platform and any side obstacles. Without this bumper, the finished vehicle could be hard to manoeuvre out of tight spots.
\subsection{RBR module}
To modularize the construction, we chose that the RBRs should be self-contained with drive assembly and wave baffles in a module. This has the advantage of lowering the weight of the platform and reducing the transport dimensions. It also allowed development and initial testing of the drive assembly and measurements of power consumption without a finished floating platform. Furthermore, as the operating depth of the RBRs was one of the parameters needed to be determined in the final tests, it was decided that the drive assembly needed to be adjustable with threaded height adjustment rods.

Thus, the complete module assembly consists of two baffles, separated by adjustable threaded rods. One baffle houses a PTFE shaft guide for the RBR axle, which is driven by a modified screwdriver mounted in the other baffle. The choice to use a screwdriver was to reduce cost, as it provides a suitably strong DC-motor together with a rugged chuck which is insensitive to off-axial loads (non-axial forces like roll forces, and forces from a not perfectly lined up construction). It also speeds up disassembly of the RBR axle from the drive assembly, which simplifies the removal of the RBR from the axle. Both baffles, with shaft guide and drive assembly is mounted above the surface, with only the RBR axle and RBR protruding below the surface. To reduce the formations of vortices induced by the rotation of the RBR, a wave baffle was mounted along the RBR axle to stop air from reaching the RBR. To attach the module to the platform, it is positioned in one of the the intended slots and attached with four fly nuts and wired up to the main power supply described in the electronics section. While operating the vehicle, the rotating speed of the RBR motors can be seamlessly adjusted via remote control up to around 500 rpm
\subsection{Propulsion}
At an early stage of development, it was chosen in consultation with an RC-expert that we should use air-propellers for propulsion, as it is better than water propellers at lower speeds. The advantages of this is simplicity of construction, cost effectiveness, and that water-electricity contact concerns are more easily managed. Furthermore air propellers will not stir up the bottom of a lake, and thus not affect the water cleaning properties of the vehicle.
Disadvantages of water propellers, such as motors intended for electric RC-boats, are that they have poor low speed maneuverability. Disadvantages of air-propulsion are the bulkier form factor, as bigger rudders and propellers are needed to propel the vehicle. There is also a safety concern as air propellers are more likely to get in contact with hands, hair, and other vulnerable body parts. To prevent some of these possible incidents to happen, an acrylic glass frame were installed around the propellers.

We chose to use two air propellers mounted in the rear of the platform together with two large rudders controlled by a single servo, the control of which is described in the controller section. The complete motor-rudder assembly was laser cut from acrylic glass which is a cheap, low weight material. However, the assembly proved to be a bit fragile, and some care was needed when handling the floating platform, to prevent damaging any of the delicate parts. Schematics and design plans can be found in the appendix.
\subsection{Water protection}
To protect the RBR module drive assembly, the batteries, engine control units, and other electronic equipment, acrylic protection covers were manufactured. These does not protect the equipment from submersion, only from small splashes or rain. 
\subsection{Transportation}
To ease transportation, the weight of the platform can be decreased by first removing the power cables to the batteries, and RBR modules, both situated underneath the protective covers. The batteries and RBR modules can then be removed by releasing the battery straps and module fly nuts which allows for removing these components. Care needs to be taken such that the propulsion mounts are not damaged during transport, as these parts are fragile.

