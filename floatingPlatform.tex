\section{Floating Platform}
The floating platform is the main piece of the vehicle that has been built. It's basically a baffle of gluglam wood that has two pieces of XPS cell foam as pontoons which all other components are attached to. It can keep a mass of around 20kg above water with some margins. In the middle of the floating platform two symmetrical holes with fittings has been made in which the RBR modules can be attached or detached from. In the front of the floating platform a large rounded front bumper helps with deflection of head on collisions and adds a distance between the platform and any side obstacles, which helps with maneuvering in tight spots. The back the propulsion devices has been attached, and they cannot easily be detached. In the middle of the floating platform the majority of the electronics has been placed, including the batteries. The electronics and the RBR modules also have acrylic protection covers, protecting the equipment from small splashes of water and rain.           

\subsection{RBR module}
The RBR module consists of two baffles, separated by three adjustable threaded rods. The higher mounted baffle has a modified hand-held screwdriver attached on it, which is used to attach and drive a RBR with a shaft. The lower mounted baffle a PTFE shaft guide for a RBR axle. The lower mounted baffle also has an additional baffle that goes down into the water just above where a mounted RBR will be. This is so that no large vortices will be created above the RBR. 

The higher mounted baffle can be adjusted hight wise in order to problem dial in the depth in which a RBR will work. Usually it's enough if the RBR are on the a depth so it just passes the pontoons of the floating platform.

The speed of the RBRs can be controlled remotly through the radio controller from a low speed to up about 600 rpm on its maximum setting. Around 500 rpm is ideal, then the RBR can do its work well at the same time as the battery drain is as small as possible. 

\subsection{Propulsion}
At an early stage of development, it was chosen in consultation with an RC-expert that we should use air-propellers for propulsion, as it is better than water propellers at lower speeds. The advantages of this is simplicity of construction, cost effectiveness, and that water-electricity contact concerns are more easily managed. Furthermore air propellers will not stir up the bottom of a lake, and thus not affect the water cleaning properties of the vehicle.
Disadvantages of water propellers, such as motors intended for electric RC-boats, are that they have poor low speed maneuverability. Disadvantages of air-propulsion are the bulkier form factor, as bigger rudders and propellers are needed to propel the vehicle. There is also a safety concern as air propellers are more likely to get in contact with hands, hair, and other vulnerable body parts. To prevent some of these possible incidents to happen, an acrylic glass frame were installed around the propellers.

We chose to use two air propellers mounted in the rear of the platform together with two large rudders controlled by a single servo, the control of which is described in the controller section. The complete motor-rudder assembly was laser cut from acrylic glass which is a cheap, low weight material. However, the assembly proved to be a bit fragile, and some care was needed when handling the floating platform, to prevent damaging any of the delicate parts. Schematics and design plans can be found in the appendix.
\subsection{Water protection}
To protect the RBR module drive assembly, the batteries, engine control units, and other electronic equipment, acrylic protection covers were manufactured. These does not protect the equipment from submersion, only from small splashes or rain. 
\subsection{Transportation}
To ease transportation, the weight of the platform can be decreased by first removing the power cables to the batteries, and RBR modules, both situated underneath the protective covers. The batteries and RBR modules can then be removed by releasing the battery straps and module fly nuts which allows for removing these components. Care needs to be taken such that the propulsion mounts are not damaged during transport, as these parts are fragile.

