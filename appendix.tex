\section{Materials and parts lists}\label{sec:appendix-a}

\subsection{Floating platform}

The floating platform is built up from the following parts:

\begin{itemize}
  \item 1 main plate made from 22~mm glulam wood. The front bumper,
    servo and containing straps for the batteries are screwed to this plate. (Blueprints in \cref{fig:appendix-blueprint-main-plate}.)
  \item 4 floating elements made from 70~mm XPS cell foam. (Blueprint in \cref{fig:appendix-blueprint-ponton}.)
  \item 2 boards that goes below the floating elements (Blueprint in \cref{fig:appendix-blueprint-ponton}.)
  \item 4 250~mm M10 stainless steel threaded rods
  \item 8 70~mm M6 stainless steel threaded rods
  \item 4 M10 nuts
  \item 4 M10 lock nuts
  \item 16 M6 nuts
  \item 8 M6 wing nuts
  \item 8 M10 washers
  \item A 20~mm plastic tube bent to a quarter circle with a radius of 400~mm
  \item 2 straps to keep the batteries in place
  \item 4 carry handles
  \item Self drilling screws and washers used to attach electronics, battery
    straps, and bumper.
\end{itemize}

\subsection{Electronics}
The electronic components mounted on the platform are:
\begin{itemize}
  \item 1 Arduino Leonardo
  \item 1 Dual VNH5019 Motor Driver Shield
  \item 2 2725 Brushless out-runner motors 1600~kv
  \item 2 ESC which matches the brushless motors
  \item 2 plastic propellers
  \item 1 15~kg servo
  \item 1 R9D Radio Control Receiver
  \item 1 blue power-LED (mounted on the cover)
  \item 1 220~$\Omega$ resistor
  \item 1 power switch (mounted on the cover)
  \item 1 0.1 $\mu$F capacitor rated for at least 12~V \footnote{\label{fotnot_app} These parts can be replaced with any suitable 12~V to 5~V power supply.}
  \item 1 22 $\mu$F capacitor rated for at least 5~V\textsuperscript{\ref{fotnot_app}}
  \item 1 L4940V5 linear regulator\textsuperscript{\ref{fotnot_app}}
  \item Protoboard for the 5~V PSU\textsuperscript{\ref{fotnot_app}} (See schematic \cref{fig:appendix-circuit-diagrams} in \cref{sec:appendix-b}.)
  \item A few meters unshielded mains cable 2x1.50~mm$^2$
  \item A few meters signal cable
  \item A few sensor cables and connectors (for servo, ESC, and power to RC receiver)
  \item Heat-shrink tubing to cover connections
  \item A few blade receptacle 4.8x0.5~mm fully insulated, blade terminal red
    4.8x0.8~mm, ring cable lug 4.3~mm used to connect the cables together.
  \item A few 2.54~mm pin headers for the connections to the Arduino.
\end{itemize}
 \Cref{fig:appendix-circuit-diagrams} in \cref{sec:appendix-b} shows the complete circuit diagram of these parts. The electronics are connected according to the circuit diagrams, and the files in \texttt{RBR\_driver.zip} is loaded onto the Arduino.

\subsection{RBR modules}
Each of the two RBR modules are built from the following parts:
\begin{itemize}
  \item 1 top plate made from a 22~mm glulam wood (\cref{fig:appendix-blueprint-rbr-module})
  \item 1 bottom plate made from a 22~mm thick board (\cref{fig:appendix-blueprint-rbr-module})
  \item 1 3D printed motor mount
  \item 1 motor, shuck, and gearbox taken from a Meec Tools 12~VDC cordless drill 
  \item 3 330~mm M10 stainless steel threaded rods
  \item 3 M10 locking nuts
  \item 9 M10 nuts
  \item 12 M10 washers
  \item 2 50~mm M4 stainless steel threaded rods
  \item 2 110~mm M4 stainless steel threaded rods
  \item 10 M4 washers
  \item 10 M4 nuts
  \item 4 long M3 screws
  \item 4 M3 nuts
  \item 4 M3 washers
  \item 1 shaft guide NS29
  \item 1 piece of 50~mm wide and 200~mm long piece of 2~mm stainless steel
    bent to form a baffle.
  \item 1 piece of metal to put pressure on top of the motor.
\end{itemize}

\subsection{Covers for RBR modules and electronics}
The cover for the RBR module are made from 5 pieces of 3~mm thick acrylic glass
glued together, with a handle attached on top. The blueprints for these are shown in
\cref{fig:appendix-blueprint-rbr-cover} and \cref{fig:appendix-blueprint-electronic-cover},
both in \cref{sec:appendix-b}.

The cover for the electronics are made from 13 pieces of 3~mm thick acrylic glass
glued together. On top of this cover the power switch is connected. This cover
was fastened to the platform using handmade angle brackets and a nut.

\clearpage
\section{Blueprints and schematics}\label{sec:appendix-b}
\subsection{Circuit diagrams}
\begin{figure}[H]
  \centering
  \includegraphics[width=0.7\textwidth]{Powersupply}
  \includegraphics[width=0.7\textwidth]{circuit}
  \caption{The circuit diagram for the electronics.}
  \label{fig:appendix-circuit-diagrams}
\end{figure}

\clearpage
\subsection{Covers}
\begin{figure}[H]
  \centering
  \raisebox{-\height}{ \includegraphics[scale=1]{RBR_cover}}
  \caption{Blueprint of the RBR cover in scale 1:2.}
  \label{fig:appendix-blueprint-rbr-cover}
\end{figure}

\begin{figure}[H]
  \centering
  \raisebox{-\height}{ \includegraphics[scale=1]{Electric_cover}}
  \caption{Blueprint of the electric cover in scale 1:5.}
  \label{fig:appendix-blueprint-electronic-cover}
\end{figure}

\clearpage
\subsection{Blueprints}
\begin{figure}[H]
  \centering
   \raisebox{-\height}{\includegraphics[scale=0.4]{RBR_mod_Top}}
   \raisebox{-1.02\height}{\includegraphics[scale=0.4]{RBR_mod_Bottom}}
  \caption{The top and bottom plates of the RBR module in scale 1:5}
  \label{fig:appendix-blueprint-rbr-module}
\end{figure}

\begin{figure}[H]
    \centering
    \includegraphics[scale=0.5]{mainPlate}
    \caption{The blueprint of the main plate in scale 1:10}
    \label{fig:appendix-blueprint-main-plate}
\end{figure}

\begin{figure}[H]
    \centering
    \raisebox{-\height}{ \includegraphics[scale=0.5]{ponton}}
    \raisebox{-1.1\height}{ \includegraphics[scale=0.5]{bottom_plank}}
    \caption{The blueprint of one ponton and the bottom board in scale 1:10}
    \label{fig:appendix-blueprint-ponton}
\end{figure}

\clearpage
\section{Specification of requirements}\label{kravspec}
.\todo{Jesper}
