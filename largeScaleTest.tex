\section{Large scale test using a pH-indicator and sodium hydroxide}

\subsection{Background}
From the beginning it had been suggested by SpinChem\textsuperscript{\textregistered} that a large scale test with the finished product should take place. The test was first supposed to be conducted with water resembling that of a mining pond with all of its usual pollutants. In this volume, the vehicle should manoeuvre its way, carrying one or more RBRs, loaded with the material that performs best according to the experiments conducted during the laboratory part of the project.

However after reconsideration, SpinChem\textsuperscript{\textregistered} decided it would be less wise to create a larger volume of heavy metal polluted water. In the case that it could not be fully recovered, it would be both tedious and expensive to dispose of the polluted water. The volume for the large scale experiment was set to be 2 m$^3$.

\subsection{Method}
Instead of using polluted water it was suggested to create a pure visual test using either a coloring called allura red and purify it by loading the reactor chambers with active carbon (AC). This process colors the water red and when the colored water passes through the reactor, the allura red would get adsorbed by the AC, making it less and less red for every time a volume passes the reactor. This process takes a considerable amount of time, but in return it has the benefit of restoring the water to practically the same as it was before staining it with allura red.
 
Or using the pH-indicator phenolphthalein together with a base (NaOH) which would dye the water pink when the pH gets above 8. For the test with the phenolphthalein the reactor chambers would be loaded with cation-exchange (CER) resins amberlite. The phenolphthalein would be added to the 2 m$^3$ of water. This by itself would not induce a coloring of the water, but when adding enough NaOH to push the pH above 8-9 the phenolphthalein would dye the water into a pink color. The CER in the RBR chambers neutralizes the base until the pH drops below 8-9 and thereby decolor the water. This process would not require the full volume of water to pass through the RBR, it would only require that enough particles gets adsorbed to lower the pH below 8 in the whole volume. The concerns for the IER test were the amounts of phenolphthalein required to dye the water, since the phenolphthalein itself will not be adsorbed.

Compared to the AC process this method is very fast. However the water will not be restored after decoloring it. Since the phenolphthalein does not get adsorbed itself, it is necessary to inspect whether or not phenolphthalein would be okay to dispose of in regular sewers. Turns out phenolphthalein poses no danger as a marine pollutant\cite{url}, all the same it would be best to use as little as possible.
Due to these facts it was decided that the large scale test would be performed using the CER approach.

\subsection{Experimental set up}
A fixed volume of 150~ml distilled water, together with a fixed amount of 150~$\mu$l 1~M NaOH added for each run. A start concentration of phenolphthalein was set to $0.05$~mg/l. For each new test that concentration was halved until the pink color no longer appeared for the naked eye. The final concentration of phenolphthalein needed was a mere $0.78$~percent of the start concentration. Scaled up to the final volume of 2~m$^3$ it would only require $0.78$~g of pure phenolphthalein. The final concentration would be $1.225$~$\mu$M (see \cref{sec:calculations}).

With the known least amount of phenolphthalein required, the process of minimizing the amount of needed NaOH could start. With a fixed amount of phenolphthalein ($0.78$~percent of start concentration) in the same amount (150~ml) of distilled water, 1~Molar NaOH was added a few $\mu$l at a time until a pink color was visible. Scaling up to the final volume of 2~m$^3$ revealed that it would require $3.5$~L of 1~M NaOH for the final test.

\subsection{Distilled vs. tapwater}
Since all of the minor tests had been performed using distilled water instead of tap water, which would be used in the large scale test, there was a need to redo the previous experiments with tap water. When testing the amounts of NaOH needed, it seemed as though it required more NaOH the first run compared to the rest of the runs. Since we did not want to waste phenolphthalein, the same water was used for the following test, the only differences were that a RBR loaded with CER had been used to neutralize the NaOH in between.

The theory behind it is that the initial pH is lower than the pH where the pH-indicator turns from pink to colorless. Meaning that for the first test the initial pH is around 6-7 while the later tests starts at an initial pH of 8-9, thus requiring smaller amounts of NaOH.
Comparing the initial pH between distilled water and tapwater showed that distilled water had somewhat lower pH. After running the CER loaded RBRs without any added base, there was a slight decrease in pH in both the distilled water and tapwater. The decrease was however larger in tapwater when comparing to the start pH value.
The new test showed that it would require a small increase of phenolphthalein from $0.78$~g up to $1.17$~g. The NaOH was increased by a tenfold.

\subsection{Scaling up}
When testing in larger scales of 5~L and 70~L it became clear that the amount of base needed, in order to visually detect a pink color, decreased exponentially. When testing with 70~L it only required 20~ml instead of the expected 120~ml, which was calculated from the 150~ml experiments.

In the final volume of 2~m$^3$ it only required 150~ml of 1~M NaOH to turn the entire volume pink. This can be explained by Lamberts law which explains that there is an exponential correlation between the transmission of light through a substance, the product of the absorption and the length the light passes through the material.\cite{pierre}

\subsection{Calculations}\label{sec:calculations}
The first calculations for phenolphthalein using distilled water:
$1.17$~ml of $0.05$~mg/ml phenolphthalein was needed in a total volume of $0.150$~L:
\begin{equation}\label{eq:concentration}
    1.17 \text{ ml} \times 0.05 \text{ mg/ml}
    = 0.0585 \text{ mg} = 0.0000585 \text{ g} \text{ phenolphthalein per } 0.150 \text{ ml}.
\end{equation}

The final volume is $2000$~dm$^3$, upscaling from $0.150$~dm$^3$:
\begin{align}
x \times 0.150 \text{ dm}^3 &= 2000 \text{ dm}^3 \nonumber \\
x &= \frac{2000 \text{ dm}^3}{0.150 \text{ dm}^3} \approx 13333.33 \\
0.00117 \text{ L} \times 13333.33 &\approx 15.6 \text{ L phenolphthalein.} 
\end{align}
For the final volume of 2000~dm$^3$, there should be $15.6$~L of the $0.05$~mg/ml phenolphthalein solution.
To simplify it is practical to recalculate this into how many grams of the substance we need instead:
\begin{equation}
    15.6 \text{ L} \times 0.05 \text{ mg/ml} = 0.78 \text{ g}.
\end{equation}

To put this in concentrations for the final volume of 2000~dm$^3$:

\begin{align}
    \text{volume } V &= 2000 \text{ dm}^3 \\
    \text{molarmass phenolftalein } M_p &= 318.3228 \text{ g/mol} \\
    \text{mass } m &= 0.78 \text{ g} \\
    \text{moles } M &= \frac{0.78 \text{ g}}{318.3228 \text{ g/mol}} \approx 0.00245 \text{ mol} \\
    \text{concentration } c &= n/V = 0.000001225 \text{ mol/dm}^3 = 1.225 \mu\text{M}
\end{align}

The final amount of phenolphthalein used was calculated in the same manner as before where the only difference were that instead of $1.17$~ml of the $0.05$~mg/ml phenolphthalein solution there was $1.755$~ml.
This implies that the final concentrations increased as well but it is still in the vicinity of a few $\mu$M.

For the NaOH the calculations are not really important since most of the time it was just to pour until the water-mass started taking a pinkish color. The amounts changed drastically with every upscale performed. In the end it required 150~$\mu$l 1~M NaOH to color 2000~dm$^3$.

\subsection{Results and discussion}
After performing a few test runs, and adjusting the amount of NaOH, we managed to reduce the time needed to decolor the 2~m$^3$ water from 25 to 10~minutes. The floating platform carrying two rotating RBRs loaded with CER was manoeuvred remotely around the pool. During the test it was clearly visible how the water just beneath the platform, around the RBRs, lost its pink color.
We had guessed in advance that it would take up to 45-60 minutes to decolor the volume. The fact that it only took 10~minutes was beyond expectations.

\subsection{Conclusion}
The SpinChem\textsuperscript{\textregistered}'s patterned RBR can be put on floating constructions and be remotely manoeuvred in order to purify larger volumes of polluted water. This can be done within reasonable time frames. These tests supports the possibility to load the reactors with other materials in order to specifically adsorb certain kinds of pollutants. It has the potential to purify water without having a negative effect on the environment.
