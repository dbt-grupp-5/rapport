\section{Large scale test using a pH-indicator and sodium hydroxide}

\subsection{Background}
From the beginning it had been suggested by the company that a large scale test
with the finished product should take place. The test was first supposed to be
conducted with water resembling that of a mining pond with all of its usual
pollutants. The volume was set to be 20~m$^3$. In this volume, the raft should
maneuver its way, carrying one or more of their patterned reactors, loaded with
the material that performs best according to the experiments conducted during
the course of the project.  However after reconsideration, the company decided
it would be less wise to create 20 m$^3$ of heavy metal polluted water. In the
case that it could not be fully recovered, it would be both tedious and
expensive to dispose of. The volume decided was also deemed excessive and was
lowered to 2 m$^3$ instead.

\subsection{Method}
Instead of using polluted water it was suggested to create a pure visual test
using either a coloring called allura red and purify it by loading the reactor
chambers with active carbon (AC). This process colors the water red and when
the colored water passes through the reactor, the allura red would get adsorbed
by the AC, making it less and less red for every time a volume passes the
reactor. This process takes a considerable amount of time, but in return it has
the benefit of restoring the water to practically the same as it was before
staining it with allura red.

Or using the pH-indicator phenolftalein together with a base (NaOH) which would
dye the water pink when pH gets above~8. For the test with the phenolftalein
the reactor chambers would be loaded with cation-exchange (CER) resins
amberlight. The phenolftalein would be added to the 2~m$^3$ of water. This by
itself would not induce a coloring of the water, but when adding enough NaOH to
push the pH above 8-9 the phenolftalein would dye the water into a pink color.
The CER in the reactor chambers neutralizes the base until such that pH drops
below 8-9 and thereby decolor the water. This process would not require the
full volume of water to pass through the reactor, it would only require that
enough particles gets adsorbed to lower the pH below 8 in the whole volume.
The concerns for the IER test were the amounts of phenolftalein required to dye
the water, since the phenolftalein itself will not be adsorbed.

Compared to the AC process this method is very fast. However the water will not
be restored after decoloring it. Since the phenolftalein does not get adsorbed
itself, it is necessary to inspect whether or not phenolftalein would be ok to
dispose of in regular sewers. Turns out phenolftalien poses no danger as a
marine pollutant\todo{ref}, all the same it would be best to use as little as
possible.  Due to these facts it was decided that the large scale test would be
performed using the CER approach.

\subsection{Experimental set up}
A fixed volume of 150~ml distilled water, together with a fixed amount of
150~$\mu$l 1~M NaOH added for each run. A start concentration of phenolftalein
was set to 0.05~mg/l. For each new test that concentration was halved until the
pink coloring no longer appeared for the naked eye. The final concentration of
phenolftalien needed was a mere 0.78~percent of the start concentration. Scaled
up to the final volume of 2~m$^3$ it would only require 0.78~g of pure
phenolftalein. The final concentration would be 1.225~$\mu$M.
(\todo{ref} for calculations.)
With the known least amount of phenolftalein required the process of
minimizing the amount of NaOH needed could start. With a fixed amount of
phenolftalein (0.78~percent) in the same amount (150~ml) of distilled water,
1~Molar NaOH was added a few $\mu$l at a time until a pink color was visible.
Scaling up to the final volume of 2~m$^3$ revealed that it would require 3.5~L
of 1~M NaOH for the final test. (\todo{ref} for calculations.)

\subsection{Distilled vs. tapwater}
Since all of the minor tests had been performed using distilled water instead
of tapwater, which would be used in the large scale test, there was a need to
redo the previous experiments with tapwater. When testing the amounts of NaOH
needed, it seemed as though it required more NaOH the first run compared to the
rest of the runs. Since we did not want to waste phenolftalein, the same water
was used for the following test, the only differences were that a reactor
loaded with CER had been used to neutralize the NaOH in between. The theory
behind it is that the initial pH is lower than the pH where the pH-indicator
turns from pink to colorless. Meaning that for the first test the initial pH is
around 6-7 while the later tests starts at an initial pH of 8-9, thus requiring
smaller amounts of NaOH.  Comparing the initial pH between distilled water and
tapwater showed that distilled water had somewhat lower pH. After running the
CER loaded reactors without any added base there was a slight decrease in pH in
both the distilled water and tapwater. The decrease was however larger in
tapwater when comparing to the start pH value.  The new test showed that it
would require a small increase of phenolftalein from 0.78~g up to 1.17~g. The
NaOH was increased by a 10~fold.

\subsection{Scaling up}
When scaling up the project, testing in volumes of 5~L and 70~L it became clear
that the amount of base needed in order to visually detect a pink color
decreased exponentially. When testing with 70~L it only required 20~ml instead of
the expected 120~ml that was calculated from the 150~ml experiments. In the final
volume of 2~m$^3$ it only required 150~ml of 1~M NaOH to turn the entire volume
pink.  This can be explained by Lamberts law that explains that there is an
exponential correlation between the transmission of light through a substance,
the product of the absorption and the length the light passed through the
material.


