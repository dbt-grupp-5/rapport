\subsection{Thoughts Behind Construction and Suggested Improvements}
%\subsection{Floating Platform}
The reason why XPS was chosen is that it is cheap, offers good buoyancy, is easy to form, enduring, does not absorb water, and is easy to access in hardware and construction stores. However, it is not the most beautiful material, and could thus be exchanged for a lighter weight material like hollow aluminium. A sealed aluminium construction would be buoyant.

As for the main platform, it was constructed from glulam. The main advantages with glulam is that it is durable, cheap, recyclable, does not sink, and it offers a steady platform which is easy to work with. A disadvantage is that glulam is heavy.


%The initial design for the vehicle was a simple platform with two pontoons, one on either side, and that all other components would be mounted on top of this platform.  The pontoons are constructed from XPS cell foam, which is available in hardware and construction stores. The advantage of XPS over e.g. EPS foam, while more expensive, is that the material is denser and doesn’t absorb water. Cell foam is easy to form and work with, non-hazardous, and offers excellent buoyancy.

%As the development progressed, the load that the platform would be required to bear was roughly estimated and served as a basis for the size of the pontoons. They were estimated to displace 20 kg with some margin to the water surface. The platform itself, constructed from glulam wood, was attached to the pontoons with 10 mm threaded rods. An advantage with wood is that it doesn’t sink, is recyclable, cheap, and easy to work with using simple tools. A disadvantage with wood is that it’s heavy. A hollow aluminium construction, e.g. sealed square tubing, could also be buoyant, and offer lighter weight.

To distribute weight as evenly as possible, the RBR modules were placed at either end in the open space between the pontoons, and the batteries were placed above the center of each pontoon. The batteries were placed in a slot in the platform, above the center of each pontoon, to reduce weight, lower center of gravity and increase stability. The available space for electronics was thus limited to the middle of the platform, between the RBR modules.

To improve deflection of head on collisions, a rounded front bumper was added which also serves as a means of adding distance between the platform and any side obstacles. Without this bumper, the vehicle could be hard to manoeuvre in tight spots.

\subsubsection{RBR modules}

The main thought for the RBR modules were to have them as a separate product on the vehicle. This will not just make transportations easier, it will also simplify replacements of the reactants in the RBR. By having a set of four RBR modules it is easy to just replace the modules with used up reactants with new modules. This allows the vehicle to continue cleaning the water while the reactant is changed in the removed modules.

The reason why a screwdriver was modified and chosen as motor for the RBR is that it is cheap, provides a suitably strong DC-motor, and comes with a rugged chuck which is insensitive to off-axial loads (non-axial forces like roll forces, and forces from a not perfectly lined up construction). The chuck also makes it easier to remove the RBR axle from the drive assembly. A disadvantage of using a modified screwdriver is that a custom mount had to be manufactured. Another disadvantage was that an Arduino with motor controller shield, which in turn needed a separate 5~V line, was needed to power and control the screwdriver motors via RC. This is a somewhat complicated solution, but relatively cost effective.



%On the top baffle of the RBR module we have attached a modified screwdriver to drive and attach a RBR with a shaft. The choice to use a screwdriver was to reduce cost, as it provides a suitably strong DC-motor together with a rugged chuck which is insensitive to off-axial loads (non-axial forces like roll forces, and forces from a not perfectly lined up construction). It also speeds up disassembly of the RBR axle from the drive assembly. A disadvantage of using a modified screwdriver is that a custom mount had to be manufactured. Another disadvantage was that an Arduino with motor controller shield, which in turn needed a separate 5~V line, was needed to power and control the screwdriver motors via RC. This is a somewhat complicated solution, but relatively cost effective.


%To modularize the construction, we choosed that the RBRs should be self-contained with drive assembly and wave baffles in a module. This has the advantage of reducing the weight of the platform for transport and reducing the transport dimensions. Furthermore, the modularization gives the advantage that multipe RBR modules can be built and speed up the refill process when changing the reactant inside the RBR. If one has four RBR modules the unused modules can be preloaded and one only has to change the modules when refilling, maximizing the time the vehicle can spend cleaning the environment. It also allowed development and initial testing of the drive assembly and measurements of power consumption without a finished floating platform.

%Both baffles, with shaft guide and drive assembly is mounted above the surface, with only the RBR axle and RBR protruding below the surface. To reduce the formations of vortices induced by the rotation of the RBR, a wave baffle was mounted along the RBR axle to stop air from reaching the RBR.

\subsubsection{Propulsion}

At an early stage of development, it was chosen in consultation with an RC-expert that we should use air-propellers for propulsion, as they perform better than water propellers at lower speeds. The advantages of this is simplicity of construction, cost effectiveness, and that water-electricity contact concerns are more easily managed. Furthermore air propellers will not stir up the bottom of a lake, and thus not affect the water cleaning properties of the vehicle.

Disadvantages of water propellers, such as motors intended for electric RC-boats, are that they have poor low speed manoeuvrability. Disadvantages of air-propulsion are the bulkier form factor, as bigger rudders and propellers are needed to propel the vehicle. There is also a safety concern as air propellers are more likely to get in contact with hands, hair, and other vulnerable body parts. To prevent some of these possible incidents to happen, an acrylic plastic frame were installed around the propellers. A further improvement would be to enclose the propellers in a rigid mesh.

%We chose to use two air propellers mounted in the rear of the platform together with two large rudders controlled by a single servo, the control of which is described in the controller section. The complete motor-rudder assembly was laser cut from acrylic glass which is a cheap, low weight material. However, the assembly proved to be a bit fragile, and some care was needed when handling the floating platform, to prevent damaging any of the delicate parts. Schematics and design plans can be found in \cref{sec:appendix-a}.

\subsubsection{Water Protection}
While the covers over the RBR modules and electronics helps deflect smaller water splashes from reaching the electronics, it is probably not enough to fully protect the vehicle from heavy rain, or being submerged into the water. Should the equipment be in need of better protection, the electronic equipment could be mounted in an IP classified enclosure. This was not done because of budget restrictions.
