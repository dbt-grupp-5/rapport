\section{Thoughts Behind Construction and Suggested Improvements}
\subsection{Floating Platform}
The initial design for the vehicle was a simple platform with two pontoons, one on either side, and that all other components would be mounted on top of this platform.  The pontoons are constructed from XPS cell foam, which is available in hardware and construction stores. The advantage of XPS over e.g. EPS foam, while more expensive, is that the material is denser and doesn’t absorb water. Cell foam is easy to form and work with, non-hazardous, and offers excellent buoyancy.

As the development progressed, the load that the platform would be required to bear was roughly estimated and served as a basis for the size of the pontoons. They were estimated to displace 20 kg with some margin to the water surface. The platform itself, constructed from glulam wood, was attached to the pontoons with 10 mm threaded rods. An advantage with wood is that it doesn’t sink, is recyclable, cheap, and easy to work with using simple tools. A disadvantage with wood is that it’s heavy. A hollow aluminium construction, e.g. sealed square tubing, could also be buoyant, and offer lighter weight. 

To distribute weight as evenly as possible, the RBR modules was placed at either end in the open space between the pontoons, and the rather heavy batteries were placed above the center of each pontoon. The batteries were placed in a slot in the platform to reduce weight, lower center of gravity and increase stability. The available space for electronics was thus limited to the middle of the platform, between the RBR modules.

To improve deflection of head on collisions, a rounded front bumper was added which also serves as a means of adding distance between the platform and any side obstacles. Without this bumper, the finished vehicle could be hard to manoeuvre out of tight spots.

\subsubsection{RBR modules}
To modularize the construction, we choosed that the RBRs should be self-contained with drive assembly and wave baffles in a module. This has the advantage of lowering the weight of the platform and reducing the transport dimensions. Beyond this the modularization gives the advantage that multipe RBR modules can be built and speed up the refill process when changing the cleansing material inside the RBR. If one has four RBR modules the unused modules can be preloaded and one only has to change the modules when refilling, maximizing the time the vehicle can spend cleaning the environment. It also allowed development and initial testing of the drive assembly and measurements of power consumption without a finished floating platform. 

On the top baffle of the RBR module we have attached a modified screwdriver to drive and attach a RBR with a shaft. The choice to use a screwdriver was to reduce cost, as it provides a suitably strong DC-motor together with a rugged chuck which is insensitive to off-axial loads (non-axial forces like roll forces, and forces from a not perfectly lined up construction). It also speeds up disassembly of the RBR axle from the drive assembly. Both baffles, with shaft guide and drive assembly is mounted above the surface, with only the RBR axle and RBR protruding below the surface. To reduce the formations of vortices induced by the rotation of the RBR, a wave baffle was mounted along the RBR axle to stop air from reaching the RBR.

