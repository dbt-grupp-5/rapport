\section{Compliance to specification}

The specification of requirements, which can be seen in Appendix \ref{kravspec}, outlined several objectives of this project. How well the requirements are fulfilled can be seen in Table \ref{tab:kravspec} below.

\todo{Kolla om dimensonerna på flotten stämmer}

\FloatBarrier
\begin{table}[H]
\center
\caption{The requirements shown in \ref{kravspec}, together with how well they are fulfilled}
\begin{tabular}{l p{0.28\linewidth} | l p{0.28\linewidth}}
\toprule
\rowcolor{lightgray}
Requirement & Compliance & Requirement & Compliance\\
\midrule
1 (a) &  Not tested but assumed to function properly. & 2 (e) &  It should be easy to implement a autonomous operating system. \\
\midrule
1 (b) &  The RBRs needed to be adjustable, and are also constructed that fashion. & 2 (f) &  Not tested but assumed to not be fulfilled. \\
\midrule
1 (c) &  The constructed vehicle is 60x90 cm, this makes it larger than in the specifications. & 3 &  Most of the parts are easily replicable, and there should not be a too hard to construct a new vehicle. \\
\midrule
2 (a) &  The vehicle only use electricity as fuel. & 4 &  It is easy to remove the RBRs from the vehicle, as the RBR modules are easy to remove from the platform. Should not be a problem to change the cleaning material. \\
\midrule
2 (b) &  Can be maneuvered with a radio controller.  & 5 (a)-(e) &  All listed materials have been tested and analysed. \\
\midrule
2 (c) &  No occurring problems when the RBRs are rotating. & 6 &  A successful test was performed, with the local newspaper filming. \\
\midrule
2 (d) &  Batteries are used, which can easily be removed and charged.  & 7 &  This objective was only partially completed, through video footage from different angles at the final tests. \\
\bottomrule
\end{tabular}
\label{tab:kravspec}
\end{table}