\section{Compliance to specification}

The specification of requirements \ref{kravspec} outlined several objectives of this project. How well the requirements are fulfilled can be seen in Table \ref{tab:kravspec} below.

\FloatBarrier
\begin{table}[H]
\center
\caption{The requirements shown in \ref{kravspec}, together with how well they are fulfilled}
\begin{tabular}{l p{0.28\linewidth} l}
\toprule
\rowcolor{lightgray}
Requirement & Complience\\
\midrule
1 (a) &  Not tested but assumed to function properly\\
\midrule
1 (b) &  The RBRs needed to be adjustable, and are also constucted that fashion \\
\midrule
1 (c) &  The constructed vehicle is 60x90 cm \todo{stämmer detta?}, this makes it larger than in the specifications \\
\midrule
2 (a) &  The vehicle only use electricity as fuel \\
\midrule
2 (b) &  Can be manoeuvred with a radio controller  \\
\midrule
2 (c) &  No occuring problems when the RBRs are rotating \\
\midrule
2 (d) &  Batteries are used, which can easily be removed and charged  \\
\midrule
2 (e) &  It should be easy to implement a autonomous operating system \\
\midrule
2 (f) &  Not tested but assumed to not be fulfilled \\
\midrule
3 &  Most of the parts are easily replicable, and there should not be a too hard to construct a new vehicle \\
\midrule
4 &  It is easy to remove the RBRs from the vehicle, as the RBR-modules are easy to remove from the platform. Should not be a problem to change the cleaning material \\
\midrule
5 (a)-(e) &  All listed materials have been tested and analysed \\
\midrule
6 &  A successful test was performed, with the local newspaper filming \\
\midrule
7 &  This objective was only partially completed, through video footage from different angles at the final tests. \\
\bottomrule
\end{tabular}
\label{tab:kravspec}
\end{table}


\begin{itemize}
	\item The ability to determine when the reactant has been consumed. This has been experimentally verified, but no measurement method has been developed.
	\item The outer dimensions of the vehicle exceeded those from the specification, but because of the modularization of the vehicle, with removable RBR modules and pontoons, the size remains manageable.
	\item There is no way to measure the remaining lifetime of the batteries, but this has been experimentally tested and is known to last long enough such that the reactants are consumed faster than the batteries are depleted.
	\item One optional objective was to get the vehicle to drive autonomously. Due to time restrictions this was never started as it was considered too time consuming. 
\end{itemize}