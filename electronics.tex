\section{Electronics}

The design and implementation of the electronics which powers the vehicle revolves
around a system of two 12~V led-acid batteries connected in parallel. These are connected
to a power distributer, which in turn connects to the motor controller shield, propulsion
motor electronic speed controllers (ESCs), the R9D radio receiver,
and 12~V to 5~V adapter which supplies power to the Arduino.

The propulsion motor ESCs and air rudder servo receives their signals directly from the R9D receiver,
but as the RBR module motor assembly is a custom part it needed a different means of speed regulation.
This is the purpose of the Arduino with motor shield. The Arduino reads a channel from the receiver
and sends the appropriate current from the motor shield to the RBR motors.

The Arduino can in theory be powered by the motor shield, negating the need for a
separate 5~V power supply. However, initial tests showed power fluctuations when all components
where used simultaneously. Thus the 5~V supply was added to provide stable power.

The power to all electronic components can be switched off using the power switch on the 
electronic cover, and a power-LED indicates the power status.
Parts lists for the electronic components can be seen in \cref{sec:appendix-electronics},
and circuit diagrams can be seen in \cref{sec:appendix-b}.