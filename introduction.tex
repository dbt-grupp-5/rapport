\section{Introduction}

SpinChem has developed a rotating bed reactor (RBR) for chemical and medicinal
applications. This device is usually mounted in a stationary set-up in a
controlled environment. This report describes a method, and platform, for
mobilizing the RBR for usage in various bodies of fluid. More specifically,
there was an interest in investigating if the vehicle, together with the RBR,
could be used for cleaning polluted waters in our environment, e.g. heavy metal
polluted lakes or ponds created by the mining industry. The project hinged on
the ability to construct a floating platform which could deliver power,
mobility, and remote control for one or more RBRs. As the application was for
use in water, all electronics needed to be suitably waterproof, and overall, a
desire was for the design to be easily replicable, such that as many products as
possible are ``of the shelf'' products.

Another part of this project was to use a RBR and evaluate the capacity of
different materials to adsorb metal ions, such as Copper ions and Zinc ions, in
polluted water over time. All work was done during the course Design-Build-Test,
and the constraints were one semester, with seven engineering students working
part time, and a budget of around 10.000 SEK provided from Umeå University.
