\section{Introduction}

SpinChem\textsuperscript{\textregistered} has developed a rotating bed reactor (RBR) for chemical and medicinal
applications. This device is usually mounted in a stationary set-up in a
controlled environment. This report describes a method, and a platform, for
mobilizing the RBR for usage in various bodies of fluid. More specifically,
there was an interest in investigating if a floating vehicle, together with the RBR,
could be used for cleaning polluted waters in our environment. This can for example be heavy metal
polluted lakes or ponds created by the mining industry. The project depended on
the ability to construct a floating platform which could deliver power,
mobility, and remote control for one or more RBRs. As the application is intended for
use in water, all electronics needed to be suitably waterproof. Overall, the
desire for the design was to be easily reproducible, such that as many parts and components as
possible are ``of the shelf'' products.

Another part of this project was to use a RBR in order to evaluate the capacity of different materials to adsorb the heavy metal ions, copper and zinc, in polluted water over a defined time interval. A literature search was made on potential materials and concrete, zeolites, goethite, ion exchange resins and active carbon, were selected for this study.

The report is split into four parts. The first is simply just an instruction manual for the floating vehicle we have built, second part is a deeper look into the process of building the vehicle and recommendation for improvements in the future. The third part is all about how we testing the cleaning properties of the floating vehicle. The last part is simply just a document describing our compliance to the specification of the project. 
