\section{Introduction}

SpinChem\textsuperscript{\textregistered} has developed a rotating bed reactor (RBR) for chemical and medicinal
applications. This device is usually mounted in a stationary set-up in a
controlled environment. This report describes a method, and a platform, for
mobilizing the RBR for usage in various bodies of fluid. More specifically,
there was an interest in investigating if a floating vehicle, together with the RBR,
could be used for cleaning polluted waters in our environment. This can for example be heavy metal
polluted lakes or ponds created by the mining industry. The project depended on
the ability to construct a floating platform which could deliver power,
mobility, and remote control for one or more RBRs. As the application is intended for
use in water, all electronics needed to be suitably waterproof. Overall, the
desire for the design was to be easily reproducible, such that as many parts and components as
possible are ``of the shelf'' products.

Another part of this project was to use a RBR in order to evaluate the capacity of
different materials to adsorb specific metal ions, such as copper and zinc, in
polluted water. All work was done during the Design-Build-Test course. The constraints were one semester, where seven engineering students worked
part time, with a 10.000 SEK budget provided by Umeå University.
